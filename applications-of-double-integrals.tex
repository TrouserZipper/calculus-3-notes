\documentclass[12pt]{article}

\usepackage{amsmath}
\usepackage{amssymb}
\usepackage{graphicx}
\usepackage{tikz}

\counterwithin*{equation}{section}
\counterwithin*{equation}{subsection}
\counterwithin*{equation}{subsubsection}
\addtolength\parskip{\bigskipamount}

\graphicspath{ {./images/} } 

\begin{document}
\section{Applications of Double Integrals}
\subsection{Density and Mass}
Picture: Lamina of unknown material with functions: \\
\(\rho(x,y) = \) density at \((x,y)\) units of mass per unit area.
\[
	\rho(x,y) = \lim_{\triangle A \to 0} \frac{\triangle m}{\triangle A}
\]
\begin{itemize}
	\item We want to find the total mass \(m\) of the lamina (a thin layer, plate, or scale of sedimentary rock, organic tissue, or other material
	\item Define \(\rho(x,y)=0	\) outside of \(D\)
	\item Choose a samplepoint \((x^{*}_{ij}, y^{*}_{ij})\) in \(R_{ij}\)
	\item Use \(\triangle A = \) area of \(R_{ij} \Rightarrow \) mass of the lamina \(\approx \pi(x^{*}_{ij}, y^{*}_{ij})\triangle A\) in \(R_{ij}\)
\end{itemize}

An approximate total mass is: 
\[
	m \approx \sum_{i=1}^{k} \sum_{j = 1}^{l} \rho(x^{*}_{ij}, y^{*}_{ij})\triangle A
\]

If we take the limit as \(k, l \to \infty\), we get the definition of the integral:
\[
	m = \iint_{D}\rho(x,y)dA
\]

\subsection{Total Electric Charge}
\(\sigma (x,y) = \) charge density at \((x,y)\) units of charge (measured in coulumbs) per unit of area

Total charge \(= Q = \iint_{D}\sigma (x,y) dA	\) 
\subsubsection{Total Electric Charge Example}
\(\sigma (x,y) = xy C/m^{2}\) \\
Region D: region bounded by \(y = 1-x, y=1, and x = 1\)

\subsection{Moments and Centers of Mass}
\begin{itemize}
	\item Region \(D\) 
	\item \(\rho (x,y) = \) density at \((x,y)\)
	\item We define the moment of a particle about an axis as the product of its mass and its directed distance from the axis.
	\item The mass of \(R_{ij} \) is approximately \(\rho(x^{*}_{ij}, y^{*}_{ij}) \triangle A\)
	\item the moment of \(R_{ij}\) with respect to the \(x\)-axis is approximately:  \\
		\([\rho(x^{*}_{ij}, y^{*}_{ij})\triangle A] y^{*}_{ij}\)
\end{itemize}

\subsection{Moment of Inertia}
The moment of inertia of a particle of mass \(m\) about an axis is defined to be  \(mr^2\), where \(r\) is the distance from the particle to the to the axis.

We will extend this to a lamina with density function \(\rho(x,y)\) and occupying a region \(D\).

Moment of Inertia about the \(x\)-axis:
\[
	I_x = \lim_{m,n \to \infty} \sum_{i=1}^{m} \sum_{n=j = 1}^{n} (y^{*}_{ij})\rho(x^{*}_{ij},y^{*}_{ij})\triangle A
\]
\[
	I_x = \iint_{D}y^2 \rho(x,y) dA
\]

Moment of Inertia about the \(y\)-axis:
\[
	I_y = \iint_{D}x^2\rho(x,y)dA
\]

Moment of Inertia about the Origin:
\[
	I_0 = \iint_{D}(x^2 + y^{2})\rho(x,y)dA
\]
Note: \(I_0 = I_x + I_y\)
\subsection{Applications of Double Integrals Examples}
\subsubsection{Example 1}
The density at any point on a semicircular lamina is proportional to the distance from the center of the circle. Find the center of mass of the lamina \(\rho(x,y)\).

\subsubsection{Example 2}
Find the moments of Inertia \(I_x, I_y,\) and \(I_0\) of a homogeneous disk \(D\) with density \(\rho(x,y) = \rho ,\) centre at the origin, and radius \(a\).
\end{document}
