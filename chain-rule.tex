\documentclass[12pt]{article}

\usepackage{amsmath}
\usepackage{amssymb}
\usepackage{graphicx}
\usepackage{tikz}

\counterwithin*{equation}{section}
\counterwithin*{equation}{subsection}
\counterwithin*{equation}{subsubsection}
\addtolength\parskip{\bigskipamount}

\graphicspath{ {./images/} } 

\begin{document}
\section{Chain Rule}
\underline{Recall:} \\
If: \(y=f(x)\) and \(x=g(t)\), then 

\fbox{
	\begin{minipage}{1in}
	\[
		\frac{dy}{dt}=\frac{dy}{dx} * \frac{dx}{dt}
	\]
	\end{minipage}
}

\subsection{Functions of 1 variable}
Consider \(z=f(x,y)\) and \(x=g(t)\) and \(y=h(t)\):\\
\begin{tikzpicture}
	\draw (1.2, 2cm) node[anchor=south west] {z} -- (0, 0) node[anchor=north] {x};
	\draw (1.6, 2cm) -- (2.8, 0) node[anchor=north] {y};
	\draw (2.8, -0.4) -- (2.8,-2.4cm) node[anchor=north] {t};
	\draw (0, -0.4) -- (0,-2.4cm) node[anchor=north] {t};
\end{tikzpicture}

Which depicts \(z=f(x,y)\), \(x=g(t)\), \(y=h(t)\):
\[
	\frac{dz}{dt} = \frac{\partial f}{\partial x} \frac{dx}{dt} + \frac{\partial f}{\partial y} \frac{dy}{dt}
\]
Recall, in order for a function \(z\) to be differentiable at point \((a,b)\), \(\triangle z\) must be able to be put in the form:
\[
	\triangle z = \frac{\partial f}{\partial x} \triangle x + \frac{\partial f}{\partial y} \triangle y + \epsilon_1\triangle x + \epsilon_2\triangle y
\]
Where \(\epsilon_1 \rightarrow 0\) and \(\epsilon_2 \rightarrow 0\) as \((\triangle x, \triangle y) \rightarrow (0,0)\)
If functions \(\epsilon_1\) and \(\epsilon_2\) are not defined at \((0,0)\), we can define them to be 0 there.

Now, we divide both sides of the equation by \(\triangle t\):
\[
	\frac{\triangle z}{\triangle t} = \frac{\partial f}{\partial x} \frac{\triangle x}{\triangle t} + \frac{\partial f}{\partial y} \frac{\triangle y }{\triangle t} + \epsilon_1 \frac{\triangle x}{\triangle t} + \epsilon_2 \frac{\triangle y }{\triangle t}
\]

We show \(\triangle x = g(t+\triangle t) - g(t) \rightarrow 0\) as \(\triangle t \rightarrow 0\), and it is assumed \(g\) is differentiable. 
\\And in the same way, \(\triangle y = h(t+\triangle t) - h(t) \to 0\) as \(\triangle t \to 0\), again assuming \(h\) is differentiable near \(t\).

We know previously from Tangent Planes and Linearization that as \(\triangle x \to 0 \), \(\epsilon_1 \to 0\) and likewise for \(\triangle y\) and \(\epsilon_2\).

Proceeding from there, because \(\epsilon_1 \rightarrow 0\) and \(\epsilon_2 \rightarrow 0\), we can now say:
\begin{align}
	\frac{dz}{dt}=\lim_{\triangle t \to 0} \frac{\triangle z}{\triangle t}
\end{align}
\begin{align}
	= \frac{\partial f}{\partial x} \lim_{\triangle t \to 0} \frac{\triangle x}{\triangle t} + \frac{\partial f}{\partial y} \lim_{\triangle t \to 0} \frac{\triangle y}{\triangle t} + (\lim_{\triangle t \to 0} \epsilon_1)\lim_{\triangle t \to 0} \frac{\triangle x}{\triangle t} + (\lim_{\triangle t \to 0} \epsilon_2) \lim_{\triangle t \to 0} \frac{\triangle y}{\triangle t}
\end{align}
\begin{align}
	= \frac{\partial f}{\partial x} \frac{dx}{dt} + \frac{\partial f}{\partial y} \frac{dy}{dt}+ 0 \cdot \frac{dx}{dt}+0\cdot \frac{dy}{dt}\\
	=\frac{\partial f}{\partial x} \frac{dx}{dt}+ \frac{\partial f}{\partial y} \frac{dy}{dt}
\end{align}
Let us rewrite this for using \(\frac{\partial z}{\partial x} \) instead of \(\frac{\partial f}{\partial x} \):

\fbox{
	\begin{minipage}{1.6in}
		\[
			\frac{dz}{dt} = \frac{\partial z}{\partial x} \frac{dx}{dt} + \frac{\partial z}{\partial y} \frac{dy}{dt}
		\]
	\end{minipage}
}

\subsection{Functions of Multiple Variables}
Now let us look at the case where \(x\) and \(y\) are functions of two variables: \(x = g(s,t)\), \(y=h(s,t)\)

We simply have two differentials rather than a single one:

\noindent\fbox{%
	\begin{minipage}{1.6in}
		\[
			\frac{\partial z}{\partial s} = \frac{\partial f}{\partial x} \frac{dx}{ds}+ \frac{\partial f}{\partial y} \frac{dy}{ds}
		\]
	\end{minipage}	
}
A function of \(t\) 

\noindent\fbox{%
	\begin{minipage}{1.6in}
		\[
			\frac{\partial z}{\partial t} = \frac{\partial f}{\partial x} \frac{dx}{dt}+ \frac{\partial f}{\partial y} \frac{dy}{dt}
		\]
	\end{minipage}	
}
A function of \(s\)

\subsubsection{General Rule}
In general, it can be shown that for \(u=f(x_1,x_2,...,x_n)\), \\%
and \(x_i = g_i(t_1,t_2,...,t_m)\):
\[
	\frac{du}{dt_i} = \sum_{j=1}^{n} \frac{\partial u}{\partial x_j} \cdot \frac{dx_j}{dt_i} = \frac{\partial u}{\partial x_1} \frac{dx}{dt_i} + \frac{\partial u}{\partial x_2}\frac{dx_2}{dt_i} + \hdots + \frac{\partial u}{\partial x_n} \frac{dx_n}{dt_i} 
\].\\
for \(i=1,2,...,m\)

\subsubsection{Take the 2nd derivative of a function: \(z=f(x,y)\)}
\(z=f(x,y), x = 2rs, y = r^2 + s^2\)\\
Find \(\frac{\partial ^2 z}{\partial r^2} \)
\begin{align}
	\frac{\partial z}{\partial r} =\frac{\partial z}{\partial x} \frac{\partial x}{\partial r}  + \frac{\partial z}{\partial y} \frac{\partial y}{\partial r} \\
	= 2s \frac{\partial z}{\partial x}  + 2r \frac{\partial z}{\partial y} \\
	\frac{\partial ^2z}{\partial r^2} = \frac{\partial }{\partial r} (\frac{\partial z}{\partial r} )\\
	\nonumber \text{Chain rule + Product rule:}\\
	\frac{\partial }{\partial r} \frac{\partial z}{\partial r} = 2s[\frac{\partial ^2z}{\partial x^2} \frac{\partial x}{\partial r} + \frac{\partial ^2z}{\partial x \partial y}\frac{\partial y}{\partial r}  ] + 2 \frac{\partial z}{\partial y} + 2r[\frac{\partial ^2z}{\partial y^2} \frac{\partial y}{\partial r} + \frac{\partial ^2z}{\partial y \partial x} \frac{\partial x}{\partial r} ]\\
	= 2 \frac{\partial z}{\partial y} + 4s^2 \frac{\partial ^2z}{\partial x^2} + 8rs \frac{\partial ^2z}{\partial x \partial y} + 4r^2 \frac{\partial ^2z}{\partial y^2} 
\end{align}

\subsection{Implicit Differentiation}
\begin{itemize}
	\item Suppose we have an equation \(F(x,y) = 0\) which implicitly defines \(y\) as a function of \(x\). That is, \(y=f(x)\) where \(F(x,f(x)) = 0\), for all \(x\) in the domain of \(f\).
	\item If \(F\) is differentiable, we may use the Chain Rule on both sides of the equation \(F(x,y) = 0\)
		\[
			\frac{\partial F}{\partial x} \frac{dx}{dx} + \frac{\partial F}{\partial y} \frac{\partial y}{\partial x} = 0
		\]
\end{itemize}

\(\Rightarrow\) 
\indent \fbox{
	\begin{minipage}{1.5in}
		\[
			\frac{dy}{dx} = \frac{-\frac{\partial F}{\partial x} }{\frac{\partial F}{\partial y} } = -\frac{F_x}{F_y}
		\]
	\end{minipage}
}

Suppose now \(z\) is implicitly defined as a function \(z=f(x,y)\) by an equation \(F(x,y,z) = 0\). That is, \(F(x, y, f(x,y)) = 0\) for all \((x,y)\) in the domain of \(f\). Then if \(F\) and \(f\) are differentiable, we have two equations:
\[
	F(x,y,z) = 0
\]
\[
	\frac{\partial F}{\partial x} \frac{\partial x}{\partial x} + \frac{\partial F}{\partial y} \frac{\partial y}{\partial x} + \frac{\partial F}{\partial z} \frac{\partial z}{\partial x} = 0
\]
Assuming \(\frac{\partial F}{\partial z} \neq 0\), we solve for either \(\frac{\partial z}{\partial x}\)  or \(\frac{\partial z}{\partial y}\) depending on what we would like to find.

\fbox{
	\begin{minipage}{2in}
		\[
			\frac{\partial z}{\partial x} -\frac{\frac{\partial F}{\partial x} }{\frac{\partial F}{\partial z} } \qquad  \frac{\partial z}{\partial y} = -\frac{\frac{\partial F}{\partial y} }{\frac{\partial F}{\partial z} }
		\]
	\end{minipage}
}
\subsubsection{Implicit Differentiation Example 1}
Find \(y'\) if \(x^3 + y^3 = xy\)
\[
	F(x,y) = x^3 + y^3 - xy  = 0
\]
\[
	\therefore \frac{dy}{dx} = =\frac{F_x}{F_y} = -\frac{3x^2 - y}{3y^2 - x}
\]

\subsubsection{Implicit Differentiation Example 2}
Find \(\frac{\partial z}{\partial x} \) and \(\frac{\partial z}{\partial y} \) if \(x^4 + y^4 + z^4 + 2xyz = 1\)\\
\(F(x,y,z) = x^4 + y^4 + z^4 + 2xyz -1 = 0\)
\[
	\frac{\partial z}{\partial x}  = -\frac{F_x}{F_z}
\]
\[
 	= -\frac{4x^3 + 2yz}{4z^3 + 2xy} = -\frac{2x^3 + yz}{2z^3 + xy}
\]
\[
	\frac{\partial z}{\partial y}  = -\frac{F_y}{F_z} = -\frac{4y^3 + 2xz}{4z^3 + 2xy} = -\frac{2y^3 + xz}{2z^3 + xy}
\]
\subsection{Examples}
\subsubsection{Chain Rule Example 1}
If \(z=x^2y^3\), \(x=\cos(2t)\), and \(y=1+\sin t\), then find \(\frac{dz}{dt}\) when \(t=0\).

\subsubsection{Chain Rule Example 2}
Ideal Gas Law \(PV=8.31T\)\\
P = Pressure (kPa)\\
V = volume (L)\\
T = temperature (K)\\
The equation relates these 3 quantities for a mole of an ideal gas. How fast is the volume changing when the temperature is 300K and \underline{increasing} by 0.1K/s, and the pressure is 25kPa and \underline{decreasing} by 0.042 kPa/s?

Givens:\\
\(P = 25kPa\)\\
\(\triangle P = \frac{dP}{dt} = -0.042 kPa/s\)\\
\(T = 300K\)\\
\(\triangle T = \frac{dT}{dt} = 0.1K/s\)\\
\(V = \frac{8.31T}{P}\)

Find \(\frac{dV}{dP}\) and \(\frac{dV}{dT} \).
\begin{align}
	\frac{dV}{dP} = -\frac{8.31T}{P^2}\\
	\frac{dV}{dT} = \frac{8.31}{P}
\end{align}

Now, solve for \(\frac{dV}{dt}\)
\begin{align}
	\frac{dV}{dt} = \frac{\partial V}{\partial P} \frac{dP}{dt} + \frac{\partial V}{\partial T} \frac{dT}{dt}\\
	\frac{dV}{dt} = -\frac{8.31T}{P^2}(-0.042) + \frac{8.31}{P}(0.1)\\
	=\frac{(0.042)(8.31)(300)}{625} + \frac{8.31(0.1)}{25}\\
	\approx .2008
\end{align}
\[
	\therefore \frac{dV}{dt} \approx .2008 L/s
\].

\subsubsection{Chain Rule Example 3}
\(z=e^xcosy\),	\(x=st^3\), \(y=s^3t\)\\
Find \(\frac{dz}{ds}\) and \(\frac{dz}{dt}\).

\subsubsection{Chain Rule Example 4}
\(w=f(x,y,z)\), \(x=x(u,v)\), \(y=y(u,v)\), \(z=z(u,v)\)\\
Find \(\frac{dw}{du}\)

\subsubsection{Chain Rule Example 5}
Suppose \(u=y^4z + x^3z^2\) where \(x=ste^r\), \(y = st^2e^{-r}\), and \(z=s^2t\sin(r)\). \\
Find \(\frac{du}{dt}\) when \(r=0\), \(s=2\), and \(t=1\).

\subsubsection{Chain Rule Example 6}
If \(g(s,t) = f(s^3 - t^3, t^3 - s^3)\), and \(f\) is differentiable, show that g satisfies the equation:
\[
	t^2 \frac{dg}{ds} + s^2 \frac{dg}{dt} = 0
\]

\subsection{Exercises}
\begin{enumerate}
	\item If \(z=x^y + 3xy^4\), where \(x= \sin(2t)\) and \(y=\cos(t)\), find \(\frac{dz}{dt}\) when \(t=0\).
	\item If \(u = x^4y + y^2z^3\), where \(x=rse^t\), \(y=rs^2e^{-t}\), and \(z=r^s\sin(t)\), find \(\frac{\partial u}{\partial s} \) when \(r = 2\), \(s = 1\), and \(t= 0\).
	\item Find \(\frac{\partial z}{\partial x}\) and \(\frac{\partial z}{\partial y}\) if \(x^3 + y^3 + z^3 + 6xyz = 1\)
	\item Revisit the Ideal Gas Law: \(PV = 8.31T\). Find the rate at which the pressure is changing when the temperature is \(300K\) and increasing by \(0.1K/s\) and the volume is \(100L\) and increasing by \(0.2L/s\).
\end{enumerate}
\end{document}
