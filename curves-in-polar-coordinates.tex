\documentclass[12pt]{article}

\usepackage{amsmath}
\usepackage{amssymb}
\usepackage{graphicx}

\counterwithin*{equation}{section}
\counterwithin*{equation}{subsection}

\graphicspath{ {./images/} } 

\begin{document}
\section{Curves in Polar Coordinates}
\includegraphics{polarcoordinates}\\%
Polar coordinates are constructed by the distance from the origin $a$ (radius) and the angle $\theta$.\\%
Previously, we used \underline{Rectangular Coordinates}\\%
\underline{Conversions}\\%
$x=r\cos\theta$\\%
$y=r\sin\theta$\\%
$\tan\theta=\frac{y}{x}$

\subsection{Example Polar 1}
\underline{Ex:} Convert $(r,\theta)=(2,\frac{\pi}{3})$ to Cartesian coordinates.

\begin{align}
	x=r \cos\theta, y=r \sin\theta\\%
	x=2 \cos \frac{\pi}{3}, y=2 \sin \frac{\pi}{3}\\%
	x=1, y=\sqrt{3}
\end{align}

\subsection{Example Polar 5}
\underline{Graph:} $r=\sin2\theta$\\%
\includegraphics{flowerpolar}

\section{Parametric Equations in Polar Coordinates}
\underline{Recall:} Parametric Equations\\%
if $r=f(\theta)$, then:\\%
$x=r\cos\theta = f(\theta)\cos\theta$\\%
$y=r\sin\theta=f(\theta)\sin\theta$\\%
$\Rightarrow\ \frac{dy}{dx}=\frac{\frac{dy}{d\theta}}{\frac{dx}{d\theta}}= \frac{-f'(\theta)\sin\theta}{f'(\theta)\cos\theta}$
\subsection{Parametrc Example 1}


\subsection{Parametric Example 2}
Find  values of $\theta$ where the tangent line is horizontal or vertical.\\%
\underline{Hor:} $\frac{dy}{dx}=0=\frac{\frac{dy}{d\theta}}{\frac{dx}{d\theta}}$


\subsection{Parametric Example Indeterminate Form}
\section{Areas and Lengths in Polar Coordinates}
\underline{Area of Sector}\\%
\includegraphics{areaofsector}\\%
Area $= A$

\underline{Area of a Polar Region}\\%
$r=f(\theta)$\\%
\includegraphics{areaofpolarregion}\\%
\bigskip
Divide polar region into sections.\\%
\bigskip
\includegraphics{areaofpolarregioncut}

\subsection{Sector area Example 1}
\underline{Consider $r=\sin2\theta$} Find the area of one leaf/petal.\\%
\includegraphics{areaofpolarregionex1}
\begin{align}
	1+2=3	
\end{align}

\subsection{Sector area Example 2}
Find the area inside the circle \underline{$r=3\cos\theta$} and outside the cardiod\\\underline{$r=1+\cos\theta$} \\%
Find intersection points\\%
$3\cos\theta = 1+\cos\theta$\\%
$\cos\theta=\frac{1}{2}\rightarrow\theta=\frac{\pi}{3},\frac{-\pi}{3}$

\subsection{Sector area Example 3}
Find all points of intersection between:\\%
a) $r=\frac{1}{2}$ and $r=\sin2\theta$
\end{document}
