\documentclass[12pt]{article}

\usepackage{amsmath}
\usepackage{amssymb}
\usepackage{graphicx}
\usepackage{tikz}

\counterwithin*{equation}{section}
\counterwithin*{equation}{subsection}
\addtolength\parskip{\bigskipamount}

\graphicspath{ {./images/} } 

\begin{document}
\section{Derivative of a Parametric Function in a 3-Dimensional Space}
\subsection{Theorem 1}

\subsection{Theorem 2}
\(\vec{r} '(t) = <f'(t),g'(t),h'(t)> \) \\%
\underline{Proof:} 
\begin{align}
	\label{eq:Derivative of parametric function r(t) }
	\vec{r} '(t) =  	
\end{align}

\subsection{Derivative of a Parametric Function Example 1}
Consider \(\vec{r} (t) = te^{-t}\hat{i} + \sin(2t)\hat{j}+(1+t^3)\hat{k} \)   \\%
Find the unit tangent vector when \(t=0\) .
\begin{align}
	\vec{r}\ '(t)=(e^{-t}-te^{-t})\hat{i} + 2\cos(2t)\hat{j} + 3t^2\hat{k} 
\end{align}
\subsection{Derivative of a Parametric Function Example 2}
Let \(\vec{r} (t) = (2-t)\hat{i} + \sqrt{t}\hat{j} \) \\%
Sketch the curve, \(\vec{r} (1) \) and \(\vec{r} \ '(1) \). 
\begin{align}
	\label{eq:Example 2 - Derivative of Parametric Function}
	x=2-t \rightarrow t=2-x \\%
	\rightarrow y=\sqrt{2-x}=\sqrt{-(x-2)}	\\%
	\vec{r} (1) = <1,1> 
\end{align}

\subsection{Derivative of a Parametric Function Example 3}	
Find parametric equations for the tangent line to the curve:\\%
\(x=\sin(t),y=t,z=2\cos(t)\) at the point \((1,\frac{\pi}{2},0)\). \\%
Notice that \(y=t  \rightarrow t = \frac{\pi}{2}\).
\begin{align}
	\vec{r} (t)=<\sin(t),t,2\cos(t)>\\%
	\vec{r} \ '(t) = <\cos(t),1,-2\sin(t)> 
\end{align}

\subsection{Differentiation Rules}
\begin{enumerate}
	\item \(\frac{d}{dt}[\vec{u} (t)  +\vec{v} (t)\ = \vec{u} \ '(t) + \vec{v}\ '(t)   \) 
	\item \(\frac{d}{dt}[c \vec{u} (t)] = c \vec{u} \ '(t)  \) 
	\item \(\frac{d}{dt}[f(t) \vec{u}(t)] = f'(t)\vec{u} (t) + f(t)\vec{u} \ '(t)    \) 
	\item \(\frac{d}{dt}[\vec{u} (r)\cdot \vec{v} (t)] = \vec{u} \ '(t)\cdot \vec{v} (t)+ \vec{u} (t) \cdot \vec{v} \ '(t)       \) 
\end{enumerate}

\subsection{Proof of (4)}
\begin{align}
	\vec{u} (t)	= <f_1(t),f_2(t),f_3(t)> 
\end{align}
\subsection{Derivative of a Parametric Function Example 3}
Suppose \(|\vec{r} \ '(t)| = c\) , a constant. Show \(\vec{r} \ '(t) \) in \underline{orthogonal} to \(\vec{r} (t) \) for all \(t\).

\begin{align}
	|\vec{r} (t)|^2 = c^2\\%
	\vec{r} (t)	\cdot \vec{r} (t) = c^2\\%
	\frac{d}{dt}: \vec{r} \ '(t) \cdot \vec{r} (t) + \vec{r} (t) \cdot \vec{r} \ '(t) = 0 \\% 
	2 \vec{r} \ '(t) \cdot \vec{r} (t) = 0\\%
	\vec{r} \ '(t) \cdot \vec{r} (t) = 0
\end{align}
\(\therefore \vec{r} \ '(t) \) is orthogonal to \(\vec{r} (t) \) 

\section{Integral of a Parametric Function in a 3-Dimensional Space}
\[
	\vec{r} (t) = f(t)\hat{i} + g(t) \hat{j} + h(t) \hat{k}
\]
\[
	\int_{a}^{b} \vec{r} (t)dt = \lim_{n \to \infty} \sum_{i=1}^{n} \vec{r} (t^*_i)\triangle t
\]
\[
	=\lim_{n \to \infty} \sum_{i=1}^{n} [f(t^{\hat{i}}_i)\hat{i}+g(t^*_i)\hat{j}) + h(t^l_i)\hat{k}] % incomplete come back
\]
For indefinite integrals, C, rather than an arbitrary constant, becomes an arbitrary vector \(\vec{C}  \) 


\end{document}

