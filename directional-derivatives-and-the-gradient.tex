\documentclass[12pt]{article}

\usepackage{amsmath}
\usepackage{amssymb}
\usepackage{graphicx}
\usepackage{tikz}

\counterwithin*{equation}{section}
\counterwithin*{equation}{subsection}
\counterwithin*{equation}{subsubsection}
\addtolength\parskip{\bigskipamount}

\graphicspath{ {./images/} } 

\begin{document}

\section{Directional Derivatives}
In general, if we wish to find the derivative of a function \(z=f(x,y)\) in the direction of \(\hat{u} = <a,b,c>\), we are looking for the tangent line \(T\) to the curve \(C\) at \(P\) in the direction \(\hat{u}\).

Let us assume \(\hat{u}\) is parallel to the \(xy\) plane. The slope of a tangent line \(T\) in the direction of \(\hat{u}\) is the rate of change of \(z\) in the direction \(\hat{u}\):
\[
	z=f(x,y) \to z_0 = f(x_0, y_0)
\]

Since \(x_0\) and \(y_0\) are given, let us solve for \((x,y)\). To this, we should find 2 points: an endpoint \(K(x,y,0)\) and an origin which lies at \(z=0\) for the point \(P\), henceforth known as \(H(x_0,y_0,0)\). 

We now know the line \(\vec{KH} \) must be parallel to the vector \(\hat{u}\).
\[
	\Rightarrow \vec{KH} = h\hat{u} \text{, for some scaler \(h\)}
\]

Reiterate:
\[
	\Rightarrow <x-x_0,y-y_0> = <ha,hb>
\]

Now, we separate components to solve for \(x\) and \(y\) individually to get two equations:
\begin{align}
	\nonumber x-x_0 = ha\\
	x = x_0 + ha
\end{align}
\begin{align}
	\nonumber y - y_0 = hb\\
	y = y_0 + hb
\end{align}

Using the definition of the derivative, we can finally find the rate of change of \(z\) in the direction of \(\hat{u}\):
\begin{align}
	\lim_{h \to 0} \frac{(f(x,y) - f(x_0,y_0)}{h} = \triangle z \text{ in the direction of \(\hat{u}\)}\\
	= \lim_{h \to 0} \frac{f(x_0 + ha, y_0 + hb) - f(x_0,y_0)}{h}
\end{align}

\underline{Theorem \#3}\\
If \(f\) is a differentiable function of \(x\) and \(y\), then \(f\) has a directional derivative in the direction of any unit vector \(\vec{u}  = <a,b>\) and:
\[
	D_{\hat{u}}f(x,y) = f_x(x,y)a + f_y(x,y)b
\]
\underline{Proof:} \\
Define g(h) = \(f(x_0+ha,y_0+hb)\). Then:
\[
	g'(0) = \lim_{h \to 0} \frac{g(h) - g(0)}{h} = \lim_{h \to 0} \frac{f(x_0 + ha,y_0+hb) - f(x_0,y_0)}{h}
\]
\[
=	D_{\hat{u}}f(x_0,y_0) 
\]
Also we may write \(g(h) = f(x,y)\) where \(x=x+ha\) and \(y=y_0 +hb\). 
Using the Chain Rule yields:
\[
	g'(h) = \frac{\partial f}{\partial x} \frac{dx}{dh}+ \frac{\partial f}{\partial y} \frac{dy}{dh} = f_x(x,y)a + f_y(x,y)b
\]
Because \(x=x_0\) and \(y=y_0\) when \(h=0\),
\[
	\Rightarrow g'(0) = f_x(x_0,y_0)a + f_y(x_0,y_0)b
\]
\[
	\therefore D_{\hat{u} }f(x_0,y_0) = f_x(x_0,y_0)a + f_y(x_0,y_0)b
\]

\subsubsection{Directional Derivatives Example 1}
\(f(x,y) = x^3 - 3xy + 4y^2\) and \(\hat{u}\) is given by \(\theta = \frac{\pi}{3}\)\\
Find \(D_{\hat{u}}f(2,1)\)

\section{Gradient}
The directional derivative of a function \(z=f(x,y)\) is \(D_{\hat{u}}f(x,y) = \triangledown f(x,y) \cdot \vec{u} \)
\[
	\triangledown f(x,y) = <f_x(x,y),f_y(x,y)>
\]
\(\triangledown f(x,y)\) is known as the gradient 
\end{document}

