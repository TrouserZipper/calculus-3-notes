\documentclass[12pt]{article}

\usepackage{amsmath}
\usepackage{amssymb}
\usepackage{graphicx}
\usepackage{tikz}

\counterwithin*{equation}{section}
\counterwithin*{equation}{subsection}
\counterwithin*{equation}{subsubsection}
\addtolength\parskip{\bigskipamount}

\graphicspath{ {./images/} } 

\begin{document}

\section{Directional Derivatives}
In general, if we wish to find the derivative of a function \(z=f(x,y)\) in the direction of \(\hat{u} = <a,b,c>\), we are looking for the tangent line \(T\) to the curve \(C\) at \(P\) in the direction \(\hat{u}\).

Let us assume \(\hat{u}\) is parallel to the \(xy\) plane. The slope of a tangent line \(T\) in the direction of \(\hat{u}\) is the rate of change of \(z\) in the direction \(\hat{u}\):
\[
	z=f(x,y) \to z_0 = f(x_0, y_0)
\]

Since \(x_0\) and \(y_0\) are given, let us solve for \((x,y)\). To this, we should find 2 points: an endpoint \(K(x,y,0)\) and an origin which lies at \(z=0\) for the point \(P\), henceforth known as \(H(x_0,y_0,0)\). 

We now know the line \(\vec{KH} \) must be parallel to the vector \(\hat{u}\).
\[
	\Rightarrow \vec{KH} = h\hat{u} \text{, for some scaler \(h\)}
\]

Reiterate:
\[
	\Rightarrow <x-x_0,y-y_0> = <ha,hb>
\]

Now, we separate components to solve for \(x\) and \(y\) individually to get two equations:
\begin{align}
	\nonumber x-x_0 = ha\\
	x = x_0 + ha
\end{align}
\begin{align}
	\nonumber y - y_0 = hb\\
	y = y_0 + hb
\end{align}

Using the definition of the derivative, we can finally find the rate of change of \(z\) in the direction of \(\hat{u}\):
\begin{align}
	\lim_{h \to 0} \frac{(f(x,y) - f(x_0,y_0)}{h} = \triangle z \text{ in the direction of \(\hat{u}\)}\\
	= \lim_{h \to 0} \frac{f(x_0 + ha, y_0 + hb) - f(x_0,y_0)}{h}
\end{align}

\underline{Theorem \#3}\\
If \(f\) is a differentiable function of \(x\) and \(y\), then \(f\) has a directional derivative in the direction of any unit vector \(\vec{u}  = <a,b>\) and:
\[
	D_{\hat{u}}f(x,y) = f_x(x,y)a + f_y(x,y)b
\]
\underline{Proof:} \\
Define g(h) = \(f(x_0+ha,y_0+hb)\). Then:
\[
	g'(0) = \lim_{h \to 0} \frac{g(h) - g(0)}{h} = \lim_{h \to 0} \frac{f(x_0 + ha,y_0+hb) - f(x_0,y_0)}{h}
\]
\[
=	D_{\hat{u}}f(x_0,y_0) 
\]
We may write \(g(h) = f(x,y)\) where \(x=x+ha\) and \(y=y_0 +hb\). 
Using the Chain Rule yields:
\[
	g'(h) = \frac{\partial f}{\partial x} \frac{dx}{dh}+ \frac{\partial f}{\partial y} \frac{dy}{dh} = f_x(x,y)a + f_y(x,y)b
\]
Because \(x=x_0\) and \(y=y_0\) when \(h=0\),
\[
	\Rightarrow g'(0) = f_x(x_0,y_0)a + f_y(x_0,y_0)b
\]
\[
	\therefore D_{\hat{u} }f(x_0,y_0) = f_x(x_0,y_0)a + f_y(x_0,y_0)b
\]


\section{Gradient}
The directional derivative of a function \(z=f(x,y)\) is \(D_{\hat{u}}f(x,y) = \triangledown f(x,y) \cdot \vec{u} \)
as demonstrated below:
\begin{align*}
	D_{\hat{u}}f(x,y) = f_x(x,y)a + f_y(x,y)b \text{, } \hat{u} = <a,b>\\
	= <f_x(x,y), f_y(x,y)> \cdot <a,b>
\end{align*}

And so: 
\[
	\triangledown f(x,y) = <f_x(x,y),f_y(x,y)>
\]
\(\triangledown f(x,y)\) is known as the gradient 

\subsubsection{Gradiant Example 1}
\(f(x,y) = cosx + e^{-xy}\)\\ 
Find \(\triangledown f(x,y)\).

\subsection{Functions of Three Variables}
Definition:
\[
	D_{\hat{u}}f(x_0,y_0,z_0) = \lim_{h \to 0} \frac{f(x_0+ha,y_0+hb,z_0+hc) - f(x_0,y_0,z_0)}{h}
\]
where \(\hat{u} = <a,b,c>\)

Another definition:
\[
	D_{\hat{u}}f(\vec{x_0}) = \lim_{h \to 0} \frac{f(\vec{x_0} + h\hat{u}) - f(\vec{x_0} )}{h}
\]

And yet another definition:
\[
	D_{\hat{u}}f(x,y,z) = \triangledown f(x,y,z) \cdot \hat{u}
\]

For a vector \(\vec{x_0} \)

\subsubsection{Functions of Three Variables Example 1}
\(f(x,y,z) = y\sin{xz}\), \(\vec{v} = 2\hat{i} - \hat{j} + \hat{k}\)
Find \(D_{\hat{u}}f(1,3,0) \) in the direction of \(\vec{v} \).

\subsection{Direction of Maximum rate of Change}
Theorem to find maximum \(\hat{u}\):
\[
	D_{\hat{u}}f(\vec{x} ) = |\triangledown f(\vec{x} )|
\]

Recall Dot Product:
\[
	\vec{v} \cdot \vec{k} = |\vec{v} | |\vec{k} | \cos{\theta}	
\]
where \(\theta\) is the angle between the two vectors.

Now:
\begin{align*}
	D_{\hat{u}}f = \triangledown f \cdot \hat{u}\\
	= |\triangledown f| |\hat{u}| \cos{\theta}\\
	= |\triangledown f| \cos{\theta} \leq |\triangledown f|
\end{align*}
The last line due to max \(\cos{\theta} = 1\) and occurs when \(\theta = 0\)

\underline{IDEA:} \(\vec{u} \text{ and }\triangledown f\) have the same direction when \(D_{\hat{u}}f\) is maximized.

\subsubsection{Direction of Maximum rate of Change Example 1}
\(f(x,y) = ye^x\), \(P(0,2)\), \(Q(2,\frac{1}{2})\)
\begin{enumerate}
	\item Find the rate of change of \(f\) at \(P\) in the direction \(\vec{PQ} \):
	\item In what direction does \(f\) have the maximum rate of change at \(P\)?
	\item What is the maximum rate of change?
\end{enumerate}

\subsection{The Normal Vector to the Tangent Plane}
\begin{itemize}
	\item Suppose we have a surface \(S\) with equation \(F(x,y,z) = k\) where \(S\) is a level surface of \(F\)
	\item Let \(P(x_0,y_0,z_0)\) be a point on \(S\) and \(C\) be any curve of \(S\) passing through \(P\)
	\item Suppose we have \(\vec{r} (t) = <x(t),y(t),z(t)>\)to describe \(C\), and \(t_0\) satisfies \(\vec{r} (t_0) = <x_0,y_0,z_0>\)
	\item Assuming \(\vec{r} (t)\) and \(F\) are differentiable, then using the chain rule yields:
		\[
			\frac{\partial F}{\partial x} \frac{dx}{dt} + \frac{\partial F}{\partial y} \frac{dy}{dt} + \frac{\partial F}{\partial z} \frac{dz}{dt} = 0
		\]
	\item Thus: 
		\[
			<F_x, F_y, F_z> \cdot <x'(t), y'(t), z'(t)> = 0
		\]
\end{itemize}
Proving that the gradiant vector of \(F\) at \(P\) is orthogonal to the tangent vector \(\vec{r}\ '(t_0) \) to any curve \(C\) on \(S\) through \(P\).

The equation of a tangent plane to \(F(x,y,z) = k\) at \(P(x_0,y_0,z_0)\):
\[
	F_x(x_0,y_0,z_0) (x-x_0) + F_y(x_0,y_0,z_0) (y-y_0) + F_z(x_0,y_0,z_0) (z-z_0) = 0
\]

The normal line to \(S\) at \(P\) is the line through \(P\) that is perpendicular to the tangent plane.

\underline{Recall:} Symmetric Equations:
\[
	\frac{x-x_0}{a} = \frac{y-y_0}{b} = \frac{z-z_0}{c}
\]
Where \(\vec{v}  = <a,b,c>\)\\
For the normal line, these are:
\[
	\frac{x-x_0}{F_x} = \frac{y-y_0}{F_y} = \frac{z-z_0}{F_z}
\]
Where \(\vec{v}  = \triangledown F(x_0,y_0,z_0)\)

\underline{Recall:} We want to use a function \(z = f(x,y)\) to compute directional derivatives\\
If \(z = f(x,y)\), then \(F(x,y,z) = f(x,y) - z = 0\)
\begin{align*}
	\Rightarrow F_x = f_x, F_y = f_y, F_z = -1\\
	f_x(x_0,y_0)(x-x_0) + f_y(x_0,y_0)(y-y_0) - (z-z_0) = 0\\
	\therefore z-z_0 = f_x(x_0,y_0)(x-x_0) + f_y(x_0,y_0)(y-y_0)
\end{align*}

\subsection{Directional Derivatives Examples}
\rule{\textwidth}{0.1mm}

\subsubsection{Directional Derivatives Example 1}
\(f(x,y) = x^3 - 3xy + 4y^2\) and \(\hat{u}\) is given by \(\theta = \frac{\pi}{3}\)\\
Find \(D_{\hat{u}}f(2,1)\)

\subsubsection{Directional Derivatives Example 2}
\(f(x,y) = x^3y^2 - 4x\), \(P(-1,2)\)\\
Find \(D_{\hat{u}}f\) at\((P)\) in the direction of \(\vec{v} = 5\hat{i} + 2\hat{j}\)

\subsubsection{Directional Derivatives Example 3}
\(S: x^2 + \frac{y^2}{9} + \frac{z^2}{4} = 3\); \(P(1,-3,-2)\)\\
Find the equations of the tangent plane and normal line to the ellipsoid \(S\) at \(P\).
\end{document}

