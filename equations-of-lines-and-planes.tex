\documentclass[12pt]{article}

\usepackage{amsmath}
\usepackage{amssymb}
\usepackage{graphicx}

\counterwithin*{equation}{section}
\counterwithin*{equation}{subsection}

\graphicspath{ {./images/} } 

\begin{document}
\section 5

\textbf{incomplete do later}

Ex. 
Find where line L intersects plane $5x-2y+4z=18$

$L: x= -4t,\ y=5+t,\ z=2+3t$
\begin{align}
5(-4t)-2(5+t)+4(2+3t)=18\\
-20t-10-2t+8+12t=18\\
-10t=20\\
t=-2
\end{align}.
\begin{enumerate}
	\item Two planes are parallel if their normal vectors are parallel.
	\item Two planes that are not parallel intersect along a line
	\item The angle between intersecting planes is the angle between their normal vectors 
\end{enumerate}

Ex.: Consider planes $x+y+z=1$ and $3x+y-2z=1$

a) Find the angle between the planes
\subsection6
\begin{align}
	\vec{n_1}=<1,1,1>,\vec{n_2}=<3,1,-2> \\
	\vec{n_1} \cdot {\vec{n_2}}=|\vec{n_1}||\vec{n_2}|\cos\theta 
\end{align}
Use the equations of two planes to describe a line\\
Distance from a point to a plane\\
\includegraphics{ramppbn}\\
$P_1(x_1,y_1,z_1)$\\
$ax+by+cz+d=0$\\


EX: Find the distance between the parallel planes

\subsection
Ex: Find the distance between the lines $L_1$ and $L_2$ 

The distance between $L_1$ and $L_2$ is teh same as teh distance between the two parallel planes that contain these lines.\\
The normal vector $\vec{n}$ for these two planes must be orthogonal to $\vec{v_1}$ and $\vec{v_2}$
\end{document}


