\documentclass[12pt]{article}

\usepackage{amsmath}
\usepackage{amssymb}
\usepackage{graphicx}
\usepackage{tikz}

\counterwithin*{equation}{section}
\counterwithin*{equation}{subsection}
\addtolength\parskip{\bigskipamount}

\graphicspath{ {./images/} } 

\begin{document}
\section{Functions of Several Variables}
\subsubsection{Functions of Several Variables Example 1}
\underline{Wave Heights} \\%
\(h=\) wave height \((m)\) \\%
\(h=f(v,t)\)\\% 
\( v=\) wind speed (km/h)\\%
\(t=\) length of time the wind has been blowing at that speed(h)

draw table for wind speed

\subsubsection{Functions of Several Variables Example 2}
\underline{Wind Chill Index}\\%
\(W\) = wind chill index (C)\\%
\(W = f(T,v)\) \\%
T = actual temperature (C)\\%
v = wind speed (km/h)

Draw table for wind chill index

The Cobb-Douglas Production Formula is a topic worth researching.

\subsubsection{Functions of Several Variables Example 3}
Solve \(1-x^2-y^2> 0\) 

\subsection{Level Curves}
\begin{itemize}
	\item The \underline{level curves} of a function in \(f\) are the curves with equations \\ \(f(x,y)=k\), where k is a constant in the range of f. 
	\item A graph of level curves is called \underline{contour map}.
\end{itemize}

For example, the contour map of a mountain range would have level curves denser when it is steeper, when \(f(x,y)\) describes the elevation for the coordinates \(x,y\).

\subsubsection{Level Curves Example 1}
Sketch the level curves of: \\%
a) \(f(x,y) = 6+2x-3y\)
\begin{quote}
	Answer here
\end{quote}
b) \(g(x,y) = x^3-y\) 
\begin{quote}
	Answer here
\end{quote}

\subsection{Functions of 3 or more Variables}
\(z=f(x_1,x_2,x_3,...,x_n) = f(\vec{x} ) \), where \(\vec{x} \in D c \mathbb{R}^n \)  
\subsubsection{Functions of 3 or more Variables Example 1}
Find the domain of \(f(x,y,z) = \sqrt{z-x^2}+y\) \\%
What does it look like?

\subsubsection{Functions of 3 or more Variablex Example 2}
What are the level surfaces of \(f(x,y,z)=x^2+3y^2+5z^2\) ?
\end{document}

