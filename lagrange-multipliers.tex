\documentclass[12pt]{article}

\usepackage{amsmath}
\usepackage{amssymb}
\usepackage{graphicx}
\usepackage{tikz}

\counterwithin*{equation}{section}
\counterwithin*{equation}{subsection}
\counterwithin*{equation}{subsubsection}
\addtolength\parskip{\bigskipamount}

\graphicspath{ {./images/} } 

\begin{document}
\section{Lagrange Multipliers}
\underline{Idea:} max/min \(f(\vec{x} ) \) \\
Subject to \(g(\vec{x} ) = k \) 

\underline{Think:} max/min \(f(x,y)\) \\
Subject to \(g(x,y) = k\) 

There exists an optimal point where they have the same tangent line at the same point.

The gradients of \(f(x,y)\) and \(g(x,y)\), \(\triangledown f\) and \(\triangledown g\),  at this optimal point are scalar multiples of each other:
\[
	\triangledown f = \lambda \triangledown g
\]
Where \(\lambda\) is known as the Lagrange Multiplier.

\subsection{Method}
To find the extreme values of \(f(x,y,z)\) subject to \(g(x,y,z) = k\) assuming they exist and \(\triangledown g \neq \vec{0}  \) on \(g(x,y,z) = k\):
\begin{enumerate}
	\item Find all values of \(x,y,z\), and \(\lambda\) such that:
		\[
			\triangledown f(x,y,z) = \lambda \triangledown g(x,y,z)
		\]
		and \(g(x,y,z) = k\) 
	\item Evaluate \(f\) at all points found in part a). Identify the extreme values.
\end{enumerate}

\subsubsection{Lagrange Multipliers Example 1}
A rectangular box without a lid is to be made from \(48m^2\) of cardboard. Find the maximum volume of such a box.

\subsubsection{Lagrange Multipliers Example 2}
\(f(x,y) = x^2 + 2y^2\); \(g(x,y) = x^2 + y^2\) \\ 
Find the extreme values of \(f\) subject to \(g(x,y) \leq 1\) 
\end{document}
