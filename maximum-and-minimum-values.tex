\documentclass[12pt]{article}

\usepackage{amsmath}
\usepackage{amssymb}
\usepackage{graphicx}
\usepackage{tikz}

\counterwithin*{equation}{section}
\counterwithin*{equation}{subsection}
\counterwithin*{equation}{subsubsection}
\addtolength\parskip{\bigskipamount}

\graphicspath{ {./images/} } 

\begin{document}
\section{Maximum and Minimum Values}
\underline{Definition:} \(f(x,y)\)  has a local max. (min.) at \((a,b)\)  \\if \(f(x,y) \leq f(a,b) (f(x,y) \geq f(a,b))\) when \((x,y)\) is near \((a,b)\).

\fbox{\(f(a,b)\) = local max.(min.) value}

\subsubsection{Fermet's Theorem \#2}
If \(f\) has a local maximum or minimum at \((a,b)\) and \(f_x(a,b)\) and \(f_y(a,b)\) exist, then \(f_x(a,b)=0\) and \(f_y(a,b) = 0\) 

\underline{Idea:} \(\triangledown f(a,b) = \vec{0}  \rightarrow \) horizontal tangent plane

\underline{Proof:} Let \(g(x) = f(x,b)\). If \(f\) has a local max.(min.) at \((a,b)\), then \(g\) has a local max.(min.) at \(a\). Thus, \(g'(a) = 0\)  by Fermet's Theorem for functions of one variable. Then:
\[
	g'(a) = f_x(a,b) = 0
\]

We can say the same for \(f_y(a,b) = 0\), where \(h(y) = f(a,y)\). If \(f\) has a local max.(min.) at \((a,b)\), then \(h\) has a local max.(min.) at \(b\). Thus, \(h'(b) = 0\) by Fermet's Theorem for functions of one variable. Then:
\[
	h'(b) = f_y(a,b) = 0
\]

\subsection{Definition of Critical Points}
A point \((a,b)\) is called a \underline{critical point} (stationary point) of \(f\)  if \(f_x(a,b) = 0\) and \(f_y(a,b) = 0\), or if one of those partial derivatives does not exist. 

By extension, if you have a local max/min at \((a,b)\), you hvae a critical point at \((a,b)\). 
However, not all critical points yield a local extreme value.

\subsection{Second Derivatives Test} 
Suppose the second partial derivatives of \(f\) are continuous on a disk with centre \((a,b)\) and \(f_x(a,b) = 0\) and \(f_y(a,b) = 0\). Now let:

\fbox{
	\begin{minipage}{3.5in}
		\[
		D = D(a,b) = f_{xx}(a,b)f_{yy}(a,b) - [f_{xy}(a,b)]^2
		\]
	\end{minipage}
}

\begin{enumerate}
	\item \(D>0\) and \(f_{xx}(a,b) > 0\) \\
		\(\Rightarrow f(a,b)\) is a local minimum
	\item \(D>0\) and \(f_{xx}(a,b) < 0\)\\
		\(\Rightarrow f(a,b)\) is a local maximum
	\item \(D<0\) \\
		\(\Rightarrow f(a,b)\) is neither a local max. nor a local min. (Saddle Point)
\end{enumerate}

\underline{Proof:} Who knows :).

\underline{Note}
\begin{itemize}
	\item \(D=0 \Rightarrow \) no information
	\item We may use a determinant \\
		\(D = \)
\end{itemize}
\subsubsection{Second Derivatives Test Example 1}
\(f(x,y) = x^4 + y^4 -2xy + 1\) \\
Find and classify the local extreme values of \(f\): 

\subsubsection{Second Derivatives Test Example 2}
Find the shortest distance from \(P(0,-2,1)\) to the plane \(2x + y + z = 4\).
\end{document}


