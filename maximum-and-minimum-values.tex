\documentclass[12pt]{article}

\usepackage{amsmath}
\usepackage{mathtools}
\usepackage{amssymb}
\usepackage{graphicx}
\usepackage{tikz}

\counterwithin*{equation}{section}
\counterwithin*{equation}{subsection}
\counterwithin*{equation}{subsubsection}
\addtolength\parskip{\bigskipamount}

\graphicspath{ {./images/} } 

\begin{document}
\section{Maximum and Minimum Values}
\underline{Definition:} \(f(x,y)\)  has a local max. (min.) at \((a,b)\)  \\if \(f(x,y) \leq f(a,b) (f(x,y) \geq f(a,b))\) when \((x,y)\) is near \((a,b)\).

\fbox{\(f(a,b)\) = local max.(min.) value}

\subsubsection{Fermet's Theorem \#2}
If \(f\) has a local maximum or minimum at \((a,b)\) and \(f_x(a,b)\) and \(f_y(a,b)\) exist, then \(f_x(a,b)=0\) and \(f_y(a,b) = 0\) 

\underline{Idea:} \(\triangledown f(a,b) = \vec{0}  \rightarrow \) horizontal tangent plane

\underline{Proof:} Let \(g(x) = f(x,b)\). If \(f\) has a local max.(min.) at \((a,b)\), then \(g\) has a local max.(min.) at \(a\). Thus, \(g'(a) = 0\)  by Fermet's Theorem for functions of one variable. Then:
\[
	g'(a) = f_x(a,b) = 0
\]

We can say the same for \(f_y(a,b) = 0\), where \(h(y) = f(a,y)\). If \(f\) has a local max.(min.) at \((a,b)\), then \(h\) has a local max.(min.) at \(b\). Thus, \(h'(b) = 0\) by Fermet's Theorem for functions of one variable. Then:
\[
	h'(b) = f_y(a,b) = 0
\]

\subsection{Definition of Critical Points}
A point \((a,b)\) is called a \underline{critical point} (stationary point) of \(f\)  if \(f_x(a,b) = 0\) and \(f_y(a,b) = 0\), or if one of those partial derivatives does not exist. 

By extension, if you have a local max/min at \((a,b)\), you have a critical point at \((a,b)\). 
However, not all critical points yield a local extreme value.

\subsection{Second Derivatives Test} 
Suppose the second partial derivatives of \(f\) are continuous on a disk with centre \((a,b)\) and \(f_x(a,b) = 0\) and \(f_y(a,b) = 0\). Now let:

\fbox{
	\begin{minipage}{3.5in}
		\[
		D = D(a,b) = f_{xx}(a,b)f_{yy}(a,b) - [f_{xy}(a,b)]^2
		\]
	\end{minipage}
}

\begin{enumerate}
	\item \(D>0\) and \(f_{xx}(a,b) > 0\) \\
		\(\Rightarrow f(a,b)\) is a local minimum
	\item \(D>0\) and \(f_{xx}(a,b) < 0\)\\
		\(\Rightarrow f(a,b)\) is a local maximum
	\item \(D<0\) \\
		\(\Rightarrow f(a,b)\) is neither a local max. nor a local min. (Saddle Point)
\end{enumerate}

\underline{Proof:} Who knows :).

\underline{Note}
\begin{itemize}
	\item \(D=0 \Rightarrow \) no information
	\item We may use a determinant: 
		\[
			 D =
			\begin{vmatrix}
				f_{xx} & f_{xy} \\
				f_{yx} & f_{yy}
			\end{vmatrix}
			= f_{xx} f_{yy} - f_{xy}^2
		\]
\end{itemize}

\subsubsection{Second Derivatives Test Example 1}
\(f(x,y) = x^4 + y^4 -2xy + 1\) \\
Find and classify the local extreme values of \(f\): 

\subsubsection{Second Derivatives Test Example 2}
Find the shortest distance from \(P(0,-2,1)\) to the plane \(2x + y + z = 4\).

\subsubsection{Second Derivatives Test Example 3}
Find the three positive numbers whose sum is \(150\) and whose product is a maximum.
\begin{align}
	\nonumber x + y + z = 150 \\
	\nonumber z = 150 - x - y \\
	f(x,y) = xy(150 - x - y) = 150xy -x^2y - xy^2 \\
	f_x(x,y) = 150y - 2xy - y^2 = y(150 - 2x - y)\\
	f_y(x,y) = 150x - x^2 - 2xy = x(150 - x - 2y)\\
	\nonumber \text{From (2): } 150 - 2x - y = 0 \\
	\nonumber \text{From (3): } x = 150 - 2y \\
	y = 150 - 2x \\
	\nonumber x = 150 - 300 + 4x \\	
	\nonumber 3x = 150 \\
	x = 50 \\
	\nonumber \text{ From (4): } y = 150-2x = 50 \\
	z = 150 - 50 - 50 = 50 
\end{align}
\fbox{
	\begin{minipage}{2in}
		\[
			\therefore x = 50, y = 50, z = 50
		\]
	\end{minipage}
}

\subsection{Extreme Value Theorem}
\underline{Review:} if \(f\) is continuous on \([a,b]\),\(\rightarrow  f\) has a max. and min. on \([a,b]\)  

Now, if \(f\) is continuous on a closed, bounded set \(D \in \mathbb{R}^2\), then \(f\)  achieves an absolute max. and min. value on \(D\).  
\begin{itemize}
	\item A \underline{closed set} in \(\mathbb{R}^2 \) is one containing all its boundary points.
	\item A \underline{boundary point} of a set \(D\)  is a point \((a,b)\) such that every disk with center \((a,b)\) contains points in and out of \(D\). 
	\item A \underline{bounded set} in \(\mathbb{R}^2\) is one that is contained within some disk.
\end{itemize}

\subsubsection{Bounded Sets Example 1}
Some closed and bounded sets
\begin{enumerate}
	\item  \(D=\{(x,y): x^2 + y^2 \leq 4\}\) 
	\item \(A = \{(x,y): |x| + |y| \leq 1\}\) 
\end{enumerate}

\subsubsection{Bounded Sets Example 2}
Closed vs Not Closed
\begin{enumerate}
	\item Closed
	\item Not closed
\end{enumerate}

\subsection{Method to Solve Extrema}
Find the absolute max/min of a continuous function \(f\) on a closed and bounded set \(D\). 
\begin{enumerate}
	\item Find \(f\) at the critical points
	\item Find the extreme values of \(f\) on the boundary of \(D\) 
	\item Identify the absolute extreme values
\end{enumerate}

\subsection{Absolute Extrema Examples}
\rule{\textwidth}{0.1mm}

\subsubsection{Example 1}
\(f(x,y) = y^2 - 2xy + 2x\), \(D = \{(x,y): 0 \leq x \leq 2, 0 \leq y \leq 3\}\) \\
Find extreme values of \(f\) on \(D\). 
\begin{align}
	f_x(x,y) = 2- 2y: y = 1 \\
	f_y(x,y) = 2y - 2x: y = x = 1\\
	\nonumber \text{Critical Point at \(f(1,1) = 1\)} \\
\end{align}
Solve for boundary points:
\begin{align*}
	f(0,0) = 0 \\
	f(0,3) = 9 \\
	f(2,0) = 4 \\
	f(2,3) = 1
\end{align*}
Check determinant at critical point:
\begin{align*}
	D = f_{xx}(x,y)f_{yy}(x,y) - f_{xy}^2(x,y) \\
	D = 0 - 2 = -2
\end{align*}
Saddle point at \(f(1,1) = 1\)

\fbox{
	\begin{minipage}{2.5in}
		Absolute max. at \((0,3,9)\) \\
		Absolute min. at \((0,0,0)\) \\
		Saddle point at \((1,1,1)\)
	\end{minipage}
}

\end{document}
