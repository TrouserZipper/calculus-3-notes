\documentclass[12pt]{article}

\usepackage{amsmath}
\usepackage{amssymb}
\usepackage{graphicx}
\usepackage{tikz}

\counterwithin*{equation}{section}
\counterwithin*{equation}{subsection}
\counterwithin*{equation}{subsubsection}
\addtolength\parskip{\bigskipamount}

\graphicspath{ {./images/} } 

\begin{document}
\section{Tangent Planes}
If two tangent lines can be acquired from a function of z, a tangent plane can be found. We have some methods by which we can form a tangent plane:

\noindent The equation of a line is given by its slope \(\frac{dy}{dx}\) multiplied by \((x-x_0)\), which is equal to \((y-y_0)\):
\[
	y-y_0 = \frac{dy}{dx}(x-x_0)
\]
Where \(x_0\) and \(y_0\) are constants representing a translation in the \(x\) and \(y\) axes respectively.

\noindent The equation of a plane is given by the standard form \(Ax + By + Cz = D\), where \(D\) is the sum of the products between \((A,B,C)\) and a given point \((x_0, y_0, z_0)\).\\
We want the equation of the plane in terms of \((x,y)\) in order to find a tangent plane to the equation \(f(x,y)= z\). So, we can set up our equation:
\begin{align}
	f(x,y) = Ax + By + Cz = D = Ax_0 + By_0 + Cz_0
\end{align}
By subtracting \(D\) which is equal to \(Ax_0 + By_0 + Cz_0\), we are left with:
\begin{align}
	A(x-x_0) + B(y-y_0) + C(z-z_0) = 0
\end{align}
Next we isolate \(z-z_0\):
\begin{align}
	z-z_0=a(x-x_0) + b(y-y_0)  
\end{align}
Where \(a=-\frac{A}{C}\) and \(b = -\frac{B}{C}\), and that is our equation of a plane in terms of \((x,y)\)\\
Now, this is beginning to resemble the equation of a line.\\
When \(y=y_0\), we see \(z-z_0 = a(x-x_0)\) which is a line where \(a\) is the slope.\\
Likewise, when \(x=x_0\), we see \(z-z_0 = b(y-y_0)\), a very similar behaviour.

\noindent Now that we have our equation of a plane in terms of \(x\) and \(y\), we can find the tangent plane to a point \(P(x_0,y_0,z_0)\) of a function \(f(x,y)\) by using the partial derivatives of \(f(x,y)\) to solve for slopes:
\[
	z-z_0 = f_x(x_0,y_0)(x-x_0) + f_y(x_0,y_0)(y-y_0)
\]

\subsubsection{Tangent Planes Example 1}
Find the tangent plane to \(z= x^2 + 2y^2\) at point \(P(1,1,3)\) 

\section{Linear Approximation}
\underline{Linearization of \(f\) at \((x_0,y_0)\) }
\[
	L(x,y) = f(x_0,y_0) + f_x(x_0,y_0)(x-x_0) + f_y(x_0,y_0)(y-y_0)
\]
\subsubsection{Non-continuous Partial Derivative}
Consider the function:
\[
	f(x,y) = \frac{xy}{x^2+y^2}
\]
is discontinuous at \(f(0,0)\) because on the line \(y=x\), \(f(x,y) = \frac{1}{2}\) on all points.

Let us take a brief detour to explore the increment of a single variable function \(y=f(x)\):\\
\(\triangle y = f(a+ \triangle x)-f(a)\)

It is possible to be shown that if \(f\) is differentiable at \(a\), then: 
\[
	\triangle y = f'(a)\triangle x + \epsilon \triangle x
\]
where \(\epsilon \rightarrow 0\) as \(\triangle x \rightarrow 0\)\\
Now, instead we can say \[
	\triangle z = f(x+\triangle x, y + \triangle y) - f(x,y)
\]
where \(z=f(x,y)\)\\
This is useful for the definition:
If \(z=f(x,y)\), then \(f\) is differentiable at \((a,b)\) if \(\triangle z\) can be expressed in the form: 
\[
	\triangle z= f_x(a,b)\triangle x + f_y(a,b)\triangle y + \epsilon_1 \triangle x + \epsilon_2\triangle y
\]
where \(\epsilon_1 \rightarrow 0\) and \(\epsilon_2 \rightarrow 0\) as \((\triangle x, \triangle y) \rightarrow (0,0)\)

\subsection{Differentials}
Recall if \(y=f(x)\) , then: \\
\(dy=f'(x)dx\) 

\subsubsection{Differentials Example 1}
Find the differential for each:\\
a) \(y=x^2\) \\
b) \(w=e^{-t}\) \\
c) \(u=\sqrt{1-x^2}\) 

\subsection{Total Differential \(dz\) }
Suppose \(z=f(x,y)\) , we define \(dx\) and \(dy\) to be independent variables\
\[
	dz = f_x(x,y)dx + f_y(x,y)dy
\]
\[
	=\frac{\partial f}{\partial x} dx + \frac{\partial f}{\partial y} dy
\]

\(dz\) can also be shown:
\[
	dz=f_x(a,b)(x-a) + f_y(a,b)(y-b)
\]
\[
	f(x,y) \approx L(x,y) = f(a,b) + dz
\]
when \((x,y)\) is near \((a,b)\) 

\subsection{Functions of Three or More Variables}
\underline{Linearization:}
\[
	f(x,y,z) \approx f(a,b,c) + f_x(a,b,c)(x-a) + f_y(a,b,c)(y-b) + f_z(a,b,c)(z-c)
\]
\underline{Differentials:} \(w=f(x,y,z)\) 
\[
	dw=\frac{dw}{dx}dx + \frac{dw}{dy}dy + \frac{dw}{dz}dz
\]
\subsubsection{Linearization Example 1}
The linearization of \(f(x,y)\) = \(x^2+2y^2\) at \(P(1,1,3)\) in \(L(x,y) = 2x+4y-3\) 

\subsubsection{Linearization Example 2}
Let \(f(x,y) = ye^{xy}\) \\
a) Show that \(f\) is differentiable at \((0,1)\) \\
b) Find the linearization of \(f\) at \((0,1)\) \\
c) Use this approximation to estimate \(f(-0.1,1.1)\) 

\subsubsection{Linearization Example 3}
Let \(z=f(x,y)=x^2-y^2\) \\
a) Find \(dz=f_x(x,y)dx+f_y(x,y)dy\)\\
b) Compute \(dz\) when \(x\) changes from \(3\) to \(2.96\) and \(y\) changes from \(2\) to \(2.05\) \\
c) Compare \(dz\) and \(\triangle z\) for the changes above.

\subsubsection{Linearization Example 4}
We measure the radius and height of of a right circular cylinder as 10cm and 25cm, respectively.
The possible error is at most 0.1cm for both measurements. Using differentials, approximate the maximum error in calculating the volume of the cylinder.

\subsubsection{Linearization Example 5}
We measure a rectangular box to be 150cm, 120cm, and 80cm to within 0.2cm for each measurement. Using differentials, estimate the largest possible error when the volume of the box is calculated.
\end{document}


