\documentclass[12pt]{article}

\usepackage{amsmath}
\usepackage{amssymb}
\usepackage{graphicx}
\usepackage{tikz}

\counterwithin*{equation}{section}
\counterwithin*{equation}{subsection}
\counterwithin*{equation}{subsubsection}
\addtolength\parskip{\bigskipamount}

\graphicspath{ {./images/} } 

\begin{document}
\section{Tangent Planes}
If two tangent lines can be acquired from a function of z 

\subsubsection{Tangent Planes Example 1}
Find the tangent plane to \(z= x^2 + 2y^2\) at point \(P(1,1,3)\) 

\section{Linear Approximation}
\underline{Linearization of \(f\) at \((x_0,y_0)\) }
\[
	L(x,y) = f(x_0,y_0) + f_x(x_0,y_0)(x-x_0) + f_y(x_0,y_0)(y-y_0)
\]

\subsection{Differentials}
Recall if \(y=f(x)\) , then: \\
\(dy=f'(x)dx\) 

\subsubsection{Differentials Example 1}
Find the differential for each:\\
a) \(y=x^2\) \\
b) \(w=e^{-t}\) \\
c) \(u=\sqrt{1-x^2}\) 

\subsection{Total Differential \(dz\) }
Suppose \(z=f(x,y)\) , we define \(dx\) and \(dy\) to be independent variables\
\[
	dz = f_x(x,y)dx + f_y(x,y)dy
\]
\[
	=\frac{\partial f}{\partial x} dx + \frac{\partial f}{\partial y} dy
\]

\(dz\) can also be shown:
\[
	dz=f_x(a,b)(x-a) + f_y(a,b)(y-b)
\]
\[
	f(x,y) \approx (L(x,y) = f(a,b) + dz
\]
when \((x,y)\) is near \((a,b)\) 

\subsection{Functions of Three or More Variables}
\underline{Linearization:}
\[
	f(x,y,z) \approx f(a,b,c) + f_x(a,b,c)(x-a) + f_y(a,b,c)(y-b) + f_z(a,b,c)(z-c)
\]
\underline{Differentials:} \(w=f(x,y,z)\) 
\[
	dw=\frac{dw}{dx}dx + \frac{dw}{dy}dy + \frac{dw}{dz}dz
\]
\subsubsection{Linearization Example 1}
The linearization of \(f(x,y)\) = \(x^2+2y^2\) at \(P(1,1,3)\) in \(L(x,y) = 2x+4y-3\) 

\subsubsection{Linearization Example 2}
Let \(f(x,y) = ye^{xy}\) \\
a) Show that \(f\) is differentiable at \((0,1)\) \\
b) Find the linearization of \(f\) at \((0,1)\) \\
c) Use this approximation to estimate \(f(-0.1,1.1)\) 

\subsubsection{Linearization Example 3}
Let \(z=f(x,y)=x^2-y^2\) \\
a) Find \(dz=f_x(x,y)dx+f_y(x,y)dy\)\\
b) Compute \(dz\) when \(x\) changes from \(3\) to \(2.96\) and \(y\) changes from \(2\) to \(2.05\) \\
c) Compare \(dz\) and \(\triangle z\) for the changes above.

\subsubsection{Linearization Example 4}
We measure the radius and height of of a right circular cylinder as 10cm and 25cm, respectively.
The possible error is at most 0.1cm for both measurements. Using differentials, approximate the maximum error in calculating the volume of the cylinder.

\subsubsection{Linearization Example 5}
We measure a rectangular box to be 150cm, 120cm, and 80cm to within 0.2cm for each measurement. Using differentials, estimate the largest possible error when the volume of the box is calculated.
\end{document}


